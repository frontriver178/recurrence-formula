\documentclass[10pt]{ltjarticle}
\usepackage[a4paper,left=15mm,right=15mm,top=15mm,bottom=20mm,footskip=10mm]{geometry}
\usepackage{luatexja,amsmath,amssymb,enumitem}
\usepackage{mhchem}
\renewcommand{\d}{\displaystyle}
\setlist[enumerate]{itemsep=1.5\baselineskip,topsep=1\baselineskip}

\title{【無機化学ガチャ】}
\date{\empty}
\author{阪大数学bot}

\begin{document}
\maketitle

\section*{問題}
次の反応の化学反応式を答えよ.

\begin{enumerate}[label=\textbf{\fbox{\arabic*}}]
  \item 単体のアルミニウムが水酸化ナトリウムに溶ける反応
  \item 硫酸銅(II)水溶液に硫化水素を通じたときの反応
  \item 塩化銅(II)水溶液に水酸化ナトリウムを加える反応
  \item 水酸化ナトリウムの固体が二酸化炭素を吸収する反応
  \item オゾンとヨウ化カリウムとの反応(ヨウ化カリウムデンプン紙→青変)
  \item 過酸化水素と酸化マンガン(IV)との反応
  \item 消石灰に塩素を吸収させる反応
\end{enumerate}

\newpage
\section*{解答}
\begin{enumerate}[label=\textbf{\fbox{\arabic*}}]
  \item \ce{2Al + 2NaOH + 6H2O -> 2Na[Al(OH)4] + 3H2}
  \item \ce{CuSO4 + H2S -> CuS + H2SO4}
  \item \ce{CuCl2 + 2NaOH -> Cu(OH)2 + 2NaCl}
  \item \ce{2NaOH + CO2 -> Na2CO3 + H2O}
  \item \ce{O3 + 2KI + H2O -> O2 + I2 + 2KOH}
  \item \ce{2H2O2 -> O2 + 2H2O}
  \item \ce{Ca(OH)2 + Cl2 -> CaCl(ClO){·}H2O}
\end{enumerate}

\end{document}