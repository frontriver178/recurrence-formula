\documentclass[10pt]{ltjarticle}
\usepackage[a4paper,
          left=15mm,
          right=15mm,
          top=15mm,
          bottom=20mm,   
          footskip=10mm  
         ]{geometry}
\usepackage{luatexja,amsmath,amssymb,enumitem}
\renewcommand{\d}{\displaystyle}
\setlist[enumerate]{%
itemsep=1.5\baselineskip, % 項目と項目の間
topsep=1\baselineskip,     % リスト直前・直後の余白
}

\title{定着度確認テスト【数列】}
\date{\today}
\author{阪大数学bot}

\begin{document}
\maketitle
学年\ \ :\\[5pt]

氏名\ \ :

\section*{問題}

次の問いに答えよ.ただし,漸化式が与えられているものについては一般項を$n$を用いて表せ.また解答例に問題番号が書かれている場合はFocusGoldの対応する問題番号の解答を参照せよ.

\begin{enumerate}[label=\textbf{\fbox{\arabic*}}]
  \item \(a_1=1,\;a_2=3,\;a_{n+2}=5a_{n+1}+6a_n\)
  \item \(a_1=4,\;a_{n+1}-a_n=-2\)
  \item \(a_1=2,\;a_{n+1}^{\frac{1}{2}}=4a_n^3\)
  \item \(a_1=2,\;a_{n+1}=a_n+6n-1\)
  \item \((x+y+z)^n\)を展開したとき、項の個数、特定項の係数の和を求めよ。
\end{enumerate}

\bigskip

\newpage
\section*{解答}
オリジナル解答解説は精神誠意作成中です.
\begin{enumerate}[label=\textbf{\fbox{\arabic*}}]
  \item 練習問題300(1)
  \item \(a_n = 6 - 2n\)
  \item 例題294
  \item \(a_n = 3n^2 - 4n + 3\)
  \item 練習307 (二項展開・項数と係数の和)
\end{enumerate}

\end{document}